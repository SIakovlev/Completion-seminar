\documentclass[11pt, aspectratio=169]{beamer}

\usepackage{subfiles}
\usepackage{float}				% To set figures position (look: http://tex.stackexchange.com/questions/8625/force-figure-placement-in-text)
\usepackage{amsmath, amssymb, amsthm, mathtools}  			% Math package
\usepackage{tikz}				% Drawing library	
\usepackage{pgfplots}
\usepackage{subfloat}			% Drawings floating package
\usepackage{algorithm}			% To write algorithms in pseudo code
\usepackage{xcolor}
\usepackage{standalone} 			%to include standalone tikz pictures
\usepackage[noend]{algpseudocode} %No end to remove endif, endfor etc.
\usepackage{framed}
\usepackage{verbatim}			% to add comments
\usepackage{savesym}
\savesymbol{Cross} 
  \usepackage{bbding}  % to add a checkmark
\restoresymbol{bb}{Cross}

\usepackage{bm}
%\usepackage{subcaption}
\usepackage{tcolorbox}
\usepackage{pifont}

% For graphs generated with matlab2tikz
\usepackage[utf8]{inputenc}
\usepackage{grffile}
\usepackage{marvosym}     % for lightning symbol

\pgfplotsset{compat=newest}
\usetikzlibrary{plotmarks}
\usepgfplotslibrary{patchplots}


\usefonttheme{structurebold}
\usefonttheme[onlymath]{serif}
\setbeamercovered{invisible}
\setbeamercolor{caption name}{fg=black}
\setbeamercolor{normal text}{bg=white,fg=black}
\setbeamercolor{frametitle}{fg=black,bg=gray!20}
\setbeamercolor{title}{bg=white,fg=black}
\setbeamercolor{section in toc}{fg=black}
\setbeamercolor{subsection in toc}{fg=black}
\setbeamertemplate{navigation symbols}{}
\setbeamertemplate{itemize item}{\color{black}$\blacktriangleright$}
\setbeamertemplate{itemize subitem}{\color{gray!50}$\blacktriangleright$}
\setbeamertemplate{itemize subsubitem}{\color{gray!20}$\blacktriangleright$}
\setbeamertemplate{footline}[frame number]

\usetikzlibrary{arrows, arrows.meta}  % Arrows library
\usetikzlibrary{calc,patterns,angles,quotes}
\usepgfplotslibrary{fillbetween}
\usepgfplotslibrary{groupplots}

\definecolor{shadecolor}{RGB}{180,180,180}

\DeclareMathOperator{\argmin}{argmin}
\DeclareMathOperator{\minimise}{minimise}
\DeclareMathOperator{\sgn}{sgn}
\DeclareMathOperator{\st}{s.t.}
\DeclareMathOperator{\diag}{diag}
\DeclareMathOperator{\dist}{dist}
\DeclareMathOperator{\Real}{Re}
\DeclareMathOperator{\Imag}{Im}

\newcommand{\cmark}{\ding{51}} % checkmark
\newcommand{\xmark}{\ding{53}} % xmark

\DeclareFontFamily{U}{matha}{\hyphenchar\font45}
\DeclareFontShape{U}{matha}{m}{n}{
      <5> <6> <7> <8> <9> <10> gen * matha
      <10.95> matha10 <12> <14.4> <17.28> <20.74> <24.88> matha12
      }{}
\DeclareSymbolFont{matha}{U}{matha}{m}{n}
\DeclareMathSymbol{\varleftarrow}{3}{matha}{"D0}
\DeclareMathSymbol{\varrightarrow}{3}{matha}{"D1}
\DeclareMathSymbol{\varuparrow}{3}{matha}{"D2}
\DeclareMathSymbol{\vardownarrow}{3}{matha}{"D3}

\tcbset{width=0.9\linewidth,boxrule=0pt,colback=gray!20,arc=0pt,auto outer arc,left=0pt,right=0pt,boxsep=3pt}

\theoremstyle{plain}
\newcommand{\backupbegin}{
   \newcounter{finalframe}
   \setcounter{finalframe}{\value{framenumber}}
}
\newcommand{\backupend}{
   \setcounter{framenumber}{\value{finalframe}}
}

% \newtheorem{lemma}{Lemma}

\setbeamercolor{postit}{fg=black,bg=yellow!5}


\title{Non-convex optimisation in antenna design and power network modelling}
\author{
Sergey Iakovlev\textsuperscript{ 1}, \\ 
Prof. Iven Mareels\textsuperscript{ 2}, \\
Prof. Rob Evans\textsuperscript{ 1}
}
\institute{
\textsuperscript{1 }Electrical and Electronic Engineering Department, The University of Melbourne, Australia, \\
\textsuperscript{2 }IBM Research, Australia.
}
\date[\today]{}

\def\HiLi{\leavevmode\rlap{\hbox to \hsize{\color{yellow!50}\leaders\hrule height .8\baselineskip depth .5ex\hfill}}} % To highlight the line

\resetcounteronoverlays{algorithm}

\tikzset{
    invisible/.style={opacity=0},
    visible on/.style={alt={#1{}{invisible}}},
    alt/.code args={<#1>#2#3}{%
      \alt<#1>{\pgfkeysalso{#2}}{\pgfkeysalso{#3}} % \pgfkeysalso doesn't change the path
    },
  }


\begin{document}

\begin{frame}[t]
\titlepage
\end{frame}

\begin{frame}[t]{Outline}
\begin{columns}[T]
\begin{column}{0.5\textwidth}

\hfill

  \only<1>{
    \tableofcontents[sections={1}]
    
    \hfill
    
    \tableofcontents[sections={2-6},sectionstyle=shaded,subsectionstyle=shaded]
  }
  \only<2>{
    \tableofcontents[sections={1},sectionstyle=shaded,subsectionstyle=shaded]
    
    \hfill
    
    \tableofcontents[sections={2}]
    
    \hfill
    
    \tableofcontents[sections={3-6},sectionstyle=shaded,subsectionstyle=shaded]
  }
  \only<3>{
    \tableofcontents[sections={1,2},sectionstyle=shaded,subsectionstyle=shaded]
    
    \hfill
    
    \tableofcontents[sections={3}]
    
    \hfill
    
    \tableofcontents[sections={4-6},sectionstyle=shaded,subsectionstyle=shaded]
  }
  \only<4>{
    \tableofcontents[sections={1-3},sectionstyle=shaded,subsectionstyle=shaded]
    
    \hfill
    
    \tableofcontents[sections={4}]
    
    \hfill
    
    \tableofcontents[sections={5-6},sectionstyle=shaded,subsectionstyle=shaded]
  }
\end{column}
\begin{column}{0.5\textwidth}

\only<1>{
\begin{itemize}
\item Main issues in LV grids
\item Motivation. Examples
\item Problem formulation. Assumptions. Phase and time synchronicity
\end{itemize}
}

\only<2>{
\begin{itemize}
\item Fixed-point iteration algorithm. Interpretation
\item Concept of decentralised calculations
\item Results
\end{itemize}
}

\only<3>{
\begin{itemize}
\item --------------------------
\item --------------------------
\end{itemize}
}

\only<4>{
\begin{itemize}
\item --------------------------
\item --------------------------
\end{itemize}
}

\end{column}
\end{columns}

\end{frame}

\section{LV grid modelling}

\subfile{LV_modelling/LV_modelling}

\section{LV grid impedance identification}

\subfile{LV_identification/LV_identification}

\section{Multi-frequency antenna beam-forming}

\subfile{Multi_frequency_antenna/Multi-frequency-antenna}

\section{Results and conclusions}

\subfile{Conclusions/Conclusions}

\section{Future work}

\end{document}

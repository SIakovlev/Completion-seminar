\documentclass[../main.tex]{subfiles}

\begin{document}


%
% -----------------------------------------------------------------------------------------------------------------
%


\begin{frame}[t]{Main issues in LV grids}
\textbf{Low-voltage} residential feeder (400 V):

\begin{center}
\begin{tikzpicture}
\draw[fill] (-1,0) circle [radius=0.1];
\draw[thick] (-1,0) -- node[above]{$\varrightarrow$} (1,0);
%\draw[fill] (1,0) circle [radius=0.1];
\only<1-4>
{
	\draw[fill] (1,0) circle [radius=0.1];
}

\uncover<7->
{
	\draw[fill] (1,0) circle [radius=0.1] node[above, black]{$\mathbf{SM}_1$};
}
\draw[arrows = {-Stealth[fill=none, inset=0pt, angle=90:10pt]}, thick] (1,0) -- node[right]{$\vardownarrow$}(1,-1);
\only<1>
{
	\draw[thick] (1,0) -- node[above]{$\varrightarrow$} (3,0);
	\draw[arrows = {-Stealth[fill=none, inset=0pt, angle=90:10pt]}, thick] (3,0) -- node[right]{$\vardownarrow$}(3,-1);
	\draw[fill] (3,0) circle [radius=0.1];
}
\only<2->
{
	\draw[thick] (1,0) -- node[above, red]{$\varleftarrow$} (3,0);
	\draw[arrows = {-Stealth[fill=none, inset=0pt, angle=90:10pt]}, ultra thick, green] (3,0) -- node[right, red]{$\varuparrow$}(3,-1);
	\draw[fill, green] (3,0) circle [radius=0.1];
}
\uncover<7->
{
	\draw[fill, green] (3,0) circle [radius=0.1] node[above, black]{$\mathbf{SM}_2$};
}

\only<1>
{
	\draw[thick] (3.1,0) -- node[above]{$\varrightarrow$} (5,0);
	\draw[arrows = {-Stealth[fill=none, inset=0pt, angle=90:10pt]}, thick] (5,0) -- node[right]{$\vardownarrow$}(5,-1);
	\draw[fill] (5,0) circle [radius=0.1];
}
\only<2->
{
	\draw[thick] (3.1,0) -- node[above, red]{$\varleftarrow$} (5,0);
	\draw[arrows = {-Stealth[fill=none, inset=0pt, angle=90:10pt]}, ultra thick, green] (5,0) -- node[right, red]{$\varuparrow$}(5,-1);
	\draw[fill, green] (5,0) circle [radius=0.1];
}
\uncover<7->
{
	\draw[fill, green] (5,0) circle [radius=0.1] node[above, black]{$\mathbf{SM}_3$};
}

\draw[thick] (5.1,0) -- (5.5,0);
\draw[thick, dotted] (5.5,0) -- (6.5,0);
\draw[thick] (6.5,0) -- (7,0);
\only<1-4>
{
	\draw[fill] (7,0) circle [radius=0.1];
}
\uncover<7->
{
	\draw[fill] (7,0) circle [radius=0.1] node[above]{$\mathbf{SM}_n$};
}
\draw[arrows = {-Stealth[fill=none, inset=0pt, angle=90:10pt]}, thick] (7,0) -- node[right]{$\vardownarrow$}(7,-1);
\end{tikzpicture}
\end{center}

\uncover<2->
{
\textbf{Issues:}
\begin{itemize}
\item<2-> Complicated and unknown model \cite{mokhtar2018automated}: 
\begin{itemize}
\item<2-> PV penetration. Backfeeding. Unpredictable behaviour of customers
\item<3-> Static information: maps, databases. All updates are manual
\item<4-> Topology errors: non-monitored switches, fuses, circuit breakers, etc
\item<5-> HV/MV grid assumptions do not work
\end{itemize}
\item<6-> Optimisation and control tools rely on the system model (impedances, topology).
\end{itemize}
}
\uncover<7->{\textbf{Possible solution:} model identification algorithms based on \textbf{smart meter data} \cite{abdel2012low}.} 
\end{frame}


%
% -----------------------------------------------------------------------------------------------------------------
%


% \begin{frame}[t]{Motivation. Examples}
% \begin{itemize}
% \item Keep network model updated
% \begin{itemize}
% \item Cables aging;
% \item Connection problems;
% \item Undocumented cable changes (after breaks, faults etc.);
% \end{itemize}
% \end{itemize}
% \end{frame}


% \begin{frame}[t, noframenumbering]{Motivation. Examples}
% \begin{itemize}
% \item Keep network model updated
% \begin{itemize}
% \item Cables aging;
% \item Connection problems;
% \item Undocumented cable changes (after breaks, faults etc.);
% \end{itemize}
% \item Fault prevention and localisation (LV cable faults change from transient to permanent);
% \begin{columns}
% \begin{column}{0.6\textwidth}
% \centering
% \begin{figure}[H]
% \begin{tikzpicture}
% \draw[dotted, thick] (0,0) -- (1,0);
% \draw[thick] (1,0) -- node[above, red]{\LARGE $\text{\Lightning}$}(6,0);
% \draw[fill] (1,0) circle [radius=0.1] node[above]{$\mathbf{SM}_{n}$};
% \draw[arrows = {-Stealth[fill=none, inset=0pt, angle=90:10pt]}, thick] (1,0) -- (1,-2);
% %\draw[arrows = {-Stealth[inset=0pt, angle=30:10pt]}, thick] (1,0) -- (1,-1) node[left]{$\mathbf{i}_{n}$};
% \draw[fill] (6,0) circle [radius=0.1] node[above]{$\mathbf{SM}_{n+1}$};
% \draw[arrows = {-Stealth[fill=none, inset=0pt, angle=90:10pt]}, thick] (6,0) -- (6,-2);
% %\draw[arrows = {-Stealth[inset=0pt, angle=30:10pt]}, thick] (6,0) -- (6,-1) node[left]{$\mathbf{i}_{n+1}$};
% \draw[dotted, thick] (6,0) -- (7,0);
% \end{tikzpicture}
% \end{figure}
% \end{column}
% \begin{column}{0.4\textwidth}
% \centering
% \begin{figure}[H]
% \begin{tikzpicture}
% \draw[fill=gray!20, draw opacity=0] (0,0.5) -- (0,1.5) -- (4,1.5) -- (4,0.5) -- cycle;
% %\draw[fill=red!20, draw opacity=0] (0,1) -- (2,1.2) -- (2,0) -- (0,0) -- cycle;
% %\draw[dashed, fill=gray!20] plot[smooth cycle] coordinates {(0,1) (2,1.2)};
% \draw[thick, ->, >=stealth] (0,0) -- (4,0) node[anchor=north west] {time};
% %\draw[dashed, fill=gray!20] plot[smooth cycle] coordinates {(0,1) (2,1.2) (2,0) (0,0)};
% %\uncover<2->{\draw[thick, dotted] (0,1) -- (4, 1)node[right]{$z_{line}$};}
% \draw[thick, black] plot [smooth] coordinates {(0,1) (2,1.2) (3,1.5) (4,2.2)}  node[left] {$z_{eff}$};
% \draw[thick, ->, >=stealth] (0,0) -- (0,2) node[left]{$z$};
% \end{tikzpicture}
% \end{figure}
% \end{column}
% \end{columns}
% \end{itemize}
% \end{frame}



% \begin{frame}[t, noframenumbering]{Motivation. Examples}
% \begin{itemize}
% \item Keep network model updated
% \begin{itemize}
% \item Cables aging;
% \item Connection problems;
% \item Undocumented cable changes (after breaks, faults etc.);
% \end{itemize}
% \item Fault prevention (LV cable faults change from transient to permanent);
% \item Stealing detection (hooking):
% \begin{columns}
% \begin{column}{0.6\textwidth}
% \centering
% \begin{figure}[H]
% \begin{tikzpicture}
% \draw[dotted, thick] (0,0) -- (1,0);
% \draw[thick] (1,0) -- (6,0);
% \draw[fill] (1,0) circle [radius=0.1] node[above]{$\mathbf{V}_{n}$};
% %\draw[arrows = {-Stealth[inset=0pt, angle=30:10pt]}, thick, visible on=<1-2>] (1,0) -- (2.5,0)  node[below]{$\mathbf{j}_{n}$};
% %\draw[arrows = {-Stealth[inset=0pt, angle=30:10pt]}, thick, visible on=<5->] (1,0) -- (2.5,0)  node[below]{$\mathbf{j}_{n}$};
% %\draw[arrows = {-Stealth[inset=0pt, angle=30:10pt]}, thick, visible on=<3-4>] (1,0) -- (2.5,0)  node[below, red]{$\mathbf{j}_{n} + \Delta\mathbf{j}_{n}$};
% \draw[arrows = {-Stealth[inset=0pt, angle=30:10pt]}, thick] (1,0) -- (2.5,0)  node[below, red]{$\mathbf{j}_{n} + \Delta\mathbf{j}_{n}$};
% \draw[arrows = {-Stealth[fill=none, inset=0pt, angle=90:10pt]}, thick] (1,0) -- (1,-2);
% %\draw[arrows = {-Stealth[inset=0pt, angle=30:10pt]}, thick] (1,0) -- (1,-1) node[left]{$\mathbf{i}_{n}$};
% %\draw[fill] (6,0) circle [radius=0.1] node[above, visible on=<1-3>]{$\mathbf{V}_{n+1}$};
% %\draw[fill] (6,0) circle [radius=0.1] node[above, visible on=<5->]{$\mathbf{V}_{n+1}$};
% %\draw[fill] (6,0) circle [radius=0.1] node[above, visible on=<4>]{$\mathbf{V}_{n+1} + \Delta\mathbf{V}_{n+1}$};
% \draw[fill] (6,0) circle [radius=0.1] node[above]{$\mathbf{V}_{n+1} + \Delta\mathbf{V}_{n+1}$};

% % theft
% %\draw[fill, visible on=<2-4>] (3.5,0) circle [radius=0.1] node[above]{Thief};
% \draw[fill] (3.5,0) circle [radius=0.1] node[above]{Thief};
% %\draw[arrows = {-Stealth[inset=0pt, angle=30:10pt]}, thick, visible on=<4>] (3.5,0) -- (5,0)  node[below, red]{$\mathbf{j}_{n}$};
% \draw[arrows = {-Stealth[inset=0pt, angle=30:10pt]}, thick] (3.5,0) -- (5,0)  node[below, red]{$\mathbf{j}_{n}$};
% %\draw[arrows = {-Stealth[fill=none, inset=0pt, angle=90:10pt]}, dotted, thick, visible on=<2-4>] (3.5,0) -- (3.5,-2);
% \draw[arrows = {-Stealth[fill=none, inset=0pt, angle=90:10pt]}, dotted, thick] (3.5,0) -- (3.5,-2);
% %\draw[arrows = {-Stealth[inset=0pt, angle=30:10pt]}, thick, visible on=<4>] (3.5,0) -- (3.5,-1) node[left, red]{$\Delta\mathbf{j}_{n}$};
% \draw[arrows = {-Stealth[inset=0pt, angle=30:10pt]}, thick] (3.5,0) -- (3.5,-1) node[left, red]{$\Delta\mathbf{j}_{n}$};

% \draw[arrows = {-Stealth[fill=none, inset=0pt, angle=90:10pt]}, thick] (6,0) -- (6,-2);
% %\draw[arrows = {-Stealth[inset=0pt, angle=30:10pt]}, thick] (6,0) -- (6,-1) node[left]{$\mathbf{i}_{n+1}$};
% \draw[dotted, thick] (6,0) -- (7,0);
% \end{tikzpicture}
% \end{figure}
% \end{column}
% \begin{column}{0.4\textwidth}
% \centering
% \begin{figure}[H]
% \begin{tikzpicture}
% \draw[fill=gray!20, draw opacity=0] (0,0.5) -- (0,1.5) -- (4,1.5) -- (4,0.5) -- cycle;
% \draw[thick,->, >=stealth] (0,0) -- (4,0) node[anchor=north west] {time};
% %\draw[black, thick] plot [smooth] coordinates {(0,1) (2,1.2) (3,2) (4,1.2)};
% % \draw[black, thick, visible on=<1-3>] plot [smooth] coordinates {(0,1) (2,1.2)} node[above left]{$z_{eff}$};
% % \draw[black, thick, visible on=<4>] plot [smooth] coordinates {(0,1) (2,1.2) (3,2)} node[left]{$z_{eff}$};
% % \draw[black, thick, visible on=<5->] plot [smooth] coordinates {(0,1) (2,1.2) (3,2) (4,1.2)} node[above right]{$z_{eff}$};
% \draw[black, thick] plot [smooth] coordinates {(0,1) (2,1.2) (3,2) (4,1.8)} node[above right]{$z_{eff}$};
% %\uncover<6->{\draw[thick, dotted] (0,1) -- (4, 1)node[right]{$z_{line}$};}
% \draw[thick,->, >=stealth] (0,0) -- (0,2) node[left]{$z$};
% \end{tikzpicture}
% \end{figure}
% \end{column}
% \end{columns}
% \end{itemize}
% \end{frame}



% \begin{frame}[t, noframenumbering]{Motivation. Examples}
% \begin{itemize}
% \item Topology identification (error correction):
% \begin{itemize}
% \item Validation tool (hypothesis testing);
% \item Combination with topology identification approaches.
% \end{itemize}
% \end{itemize}
% \begin{columns}[t]
% \begin{column}{0.6\textwidth}
% \centering
% \begin{figure}[H]
% \begin{tikzpicture}
% \draw[thick, visible on=<6->] (2,0) -- (8,0);
% \draw[thick, visible on=<6->] (6,0) -- (6,-2) -- (8,-2);
% \draw[thick, visible on=<6->] (2,0) -- (2,-2) -- (2,-4);
% \draw[thick, dotted, visible on=<4-4>] (8,-2) -- node[above]{\textbf{?}}(6,-2);
% \draw[thick, visible on=<5-5>] (8,-2) -- node[above, green]{\CheckmarkBold}(6,-2);
% \draw[thick, dotted, visible on=<2-2>] (8,-2) -- node[left]{\textbf{?}} (8,0);
% \draw[thick, dotted, visible on=<3-3>] (8,-2) -- node[left, red]{\textbf{x}} (8,0);
% \draw[fill] (2,0) circle [radius=0.1];
% \draw[fill] (4,0) circle [radius=0.1];
% \draw[fill] (6,0) circle [radius=0.1];
% \draw[fill] (8,0) circle [radius=0.1];
% \draw[fill] (6,-2) circle [radius=0.1];
% \draw[fill] (8,-2) circle [radius=0.1];
% \draw[fill] (2,0) circle [radius=0.1];
% \draw[fill] (2,-2) circle [radius=0.1];
% \draw[fill] (2,-4) circle [radius=0.1];
% \end{tikzpicture}
% \end{figure}
% \end{column}
% \begin{column}{0.4\textwidth}
% \centering
% \begin{figure}[H]
% \begin{tikzpicture}
% \draw[fill=gray!20, draw opacity=0] (0,0.5) -- (0,1.5) -- (4,1.5) -- (4,0.5) -- cycle;
% \draw[thick, ->, >=stealth] (0,0) -- (4,0) node[anchor=north west] {time};

% \draw[thick, black, visible on=<3-3>] plot [smooth] coordinates {(0,1) (1,-1) (2,1.3) (2.5, 0.5) (3, 1.5) (4,-1)};
% %\draw[thick, visible on=<3-3>] (0,0) -- (1,-1);
% %\draw[thick, visible on=<3-3>] (1,-1) -- (2,2);
% %\draw[thick, visible on=<3-3>] (2,2) -- (3,-1);


% \draw[thick, black, visible on=<5-5>] plot [smooth] coordinates {(0,1) (1,1.2) (2,1.1) (2.5, 0.9) (3, 1.1) (4,1)};
% %\draw[thick, visible on=<5-5>] (0,1) -- (4,1);

% %\draw[thick] (0,1) -- (2,1);
% %\draw[thick] (2,1) -- (2,2);
% %\draw[thick] (2,2) -- (4,2);

% \draw[thick, ->, >=stealth] (0,0) -- (0,2) node[anchor=south east] {$z_{eff}$};
% \end{tikzpicture}
% \end{figure}
% \end{column}
% \end{columns}
% \end{frame}


%
% -----------------------------------------------------------------------------------------------------------------
%

\begin{frame}[t]{Problem formulation}

\textbf{Single phase} chain distribution feeder in \textbf{steady state}:

%\setbeamercolor{normal text}{fg=gray,bg=}
\setbeamercolor{alerted text}{fg=black ,bg=yellow}

\begin{center}
\begin{tikzpicture}
\draw[fill] (0,0) circle [radius=0.1] node[above, black]{$\mathbf{V}_{0}$};
\draw[thick] (0,0) -- (9,0) node[pos=2/9, above, red]{$z_{1}$} node[pos=6/9, above, red]{$z_{2}$};
%\draw[arrows = {-Stealth[inset=0pt, angle=30:10pt]}, thick] (0,0) -- (1.5,0);%  node[below]{$\mathbf{j}_{1}$};
\draw[fill] (4,0) circle [radius=0.1] node[above]{$\mathbf{SM}_{1}$};
%\draw[arrows = {-Stealth[inset=0pt, angle=30:10pt]}, thick] (4,0) -- (5.5,0);%  node[below, red]{$\mathbf{j}_{2}$};
\draw[arrows = {-Stealth[fill=none, inset=0pt, angle=90:10pt]}, thick] (4,0) -- (4,-2);
%\draw[arrows = {-Stealth[inset=0pt, angle=30:10pt]}, thick] (4,0) -- (4,-1);% node[left]{$\mathbf{i}_{2}$};
\draw[fill] (8,0) circle [radius=0.1] node[above]{$\mathbf{SM}_{2}$};
\draw[arrows = {-Stealth[fill=none, inset=0pt, angle=90:10pt]}, thick] (8,0) -- (8,-2);
%\draw[arrows = {-Stealth[inset=0pt, angle=30:10pt]}, thick] (8,0) -- (8,-1);% node[left]{$\mathbf{i}_{3}$};
\draw[dotted, thick] (9,0) -- (10,0);
\draw[thick] (10,0) -- (11,0);
\draw[fill] (11,0) circle [radius=0.1] node[above]{$\mathbf{SM}_{N}$};
\draw[arrows = {-Stealth[fill=none, inset=0pt, angle=90:10pt]}, thick] (11,0) -- (11,-2);
%\draw[arrows = {-Stealth[inset=0pt, angle=30:10pt]}, thick] (11,0) -- (11,-1);% node[left]{$\mathbf{i}_{N}$};
\end{tikzpicture}
\end{center}

\textbf{Given:} $\mathbf{SM}_{n} = \Big[ \mathbf{V}_{n}, \bm{\theta}_{n}, \mathbf{I}_{n} \Big]$ - local smart meter measurements for each node $n$. 
\begin{itemize}
    \item $\mathbf{V}_{n}, \mathbf{I}_{n}$ - RMS voltages and currents at node $n$
    \item $\bm{\theta}_{n}$ - angles between current and voltage at node $n$.
\end{itemize}

\textbf{Find:} $\mathbin{\color{red}z_1}, \mathbin{\color{red}z_2}, \ldots, \mathbin{\color{red}z_{N-1}} \in \mathbb{C}$ \footnote{Hereinafter all unknown (from context) parameters are marked by \textcolor{red}{red} colour}.
\end{frame}



%
% -----------------------------------------------------------------------------------------------------------------
%


% \begin{frame}[t]{Related works}
% \begin{itemize}
% \item Estimation of the Thevenin equivalent grid impedance
% \begin{itemize}
% \item Many approaches: passive, active, their combinations, etc.
% \item High accuracy and good performance
% \item Limited applicability. Not suitable for:
% \begin{itemize}
% \item Fault localisation
% \item Detection
% \end{itemize}
% \end{itemize}
% \end{itemize}

% \begin{figure}[H]
% \begin{tikzpicture}
% \draw[dotted, thick] (0,0) -- (1,0);
% \draw[thick] (1,0) -- (2,0);
% \draw (2, -0.5) rectangle node[]{$z_{grid}$} (4, 0.5);
% \draw[thick] (4,0) -- (5,0) -- (5, -1);
% \draw[thick] (5, -1.5) circle [radius=0.5] node[]{$V$};
% \draw[thick] (5,-2) -- (5,-3) -- (1, -3);
% \draw[dotted, thick] (1, -3) -- (0, -3);
% \end{tikzpicture}
% \end{figure}

% \end{frame}



% \begin{frame}[t, noframenumbering]{Related works}
% \begin{itemize}
% \item Estimation of the Thevenin equivalent grid impedance
% \item Estimation of the impedance matrix
% \begin{itemize}
% \item Synchronised (PMU \footnote{Phasor Measurement Unit}, $\mu$PMU) measurements assumption:

% \only<2-3>{

% \hfill

% \hfill

% \begin{figure}[H]
% \centering
%   \begin{tikzpicture}
%     \coordinate (t0) at (0, 0.5);
%     \coordinate (tn) at (8, 0.5);
%     \draw[arrows = {-Stealth[inset=0pt, angle=25:5pt]}] (t0) -- (tn) node[above]{$t$};

%     % arrows 1
%     \coordinate (a1) at (0, 2);
%     \coordinate (c1) at (1.9, 1.7);
%     \draw[arrows = {-Stealth[inset=0pt, angle=25:5pt]}] (a1) -- (c1);
%     \draw[thin, dotted](a1) -- (0, 0.5);

%     \coordinate (a2) at (0,1.5);
%     \coordinate (c2_) at (1.9, 1.2);
%     \coordinate (c2) at (1.9, 1);
%     %\draw[dotted, thin, fill=gray!50] (a2)--(c2)--(c2_)--(a2);
%     \draw[arrows = {-Stealth[inset=0pt, angle=25:5pt]}] (a2) -- (c2_);

%     \coordinate (text1) at (1, 0);
%     \node[text width=2cm] at (text1){time: \cmark \\ phase: \cmark};

%     % arrows 2
%     \coordinate (a3) at (3,2);
%     \coordinate (c3) at (4.9,1.7);
%     \draw[arrows = {-Stealth[inset=0pt, angle=25:5pt]}] (a3) -- (c3);
%     \draw[thin, dotted](a3) -- (3, 1.5);

%     \coordinate (a4) at (3, 1.5);
%     \coordinate (a4_) at (3.2, 1.5);
%     \coordinate (c4) at (5.1, 1.2);
%     \draw[dotted, thin, fill=gray!50] (a4_)--(a4)--(3, 0.5)--(3.2, 0.5)--(a4_);
%     \draw[arrows = {-Stealth[inset=0pt, angle=25:5pt]}] (a4_) -- (c4);

%     \coordinate (text2) at (4, 0);
%     \node[text width=2cm] at (text2){time: \xmark \\ phase: \cmark};

%     % arrows 3
%     \coordinate (a5) at (6,2);
%     \coordinate (c5) at (7.9,1.7);
%     \draw[arrows = {-Stealth[inset=0pt, angle=25:5pt]}] (a5) -- (c5);
%     \draw[thin, dotted](a5) -- (6, 0.5);

%     \coordinate (a6) at (6,1.5);
%     \coordinate (c6_) at (7.9, 1.2);
%     \coordinate (c6) at (7.9, 1);
%     \draw[dotted, thin, fill=gray!50] (a6)--(c6)--(c6_)--(a6);
%     \draw[arrows = {-Stealth[inset=0pt, angle=25:5pt]}] (a6) -- (c6);

%     \coordinate (text3) at (7, 0);
%     \node[text width=2cm] at (text3){time: \cmark \\ phase: \xmark};
%   \end{tikzpicture}
% \end{figure}
% }
% \only<3>{
%   \begin{itemize}
%     \item PMU: time and phased synchronised measurements
%     \item \textbf{Proposed approach}: time synchronism only, phases are estimated during computations
%   \end{itemize}
% }
% \end{itemize}
% \end{itemize}
% \end{frame}



%
% -----------------------------------------------------------------------------------------------------------------
%



\begin{frame}[t]{Phase and time synchronism}

% Two phasors can be synchronised in \textbf{phase} and \textbf{time}:
% \begin{figure}[H]
% \centering
%   \begin{tikzpicture}
%     \coordinate (t0) at (0, 0.5);
%     \coordinate (tn) at (8, 0.5);
%     \draw[arrows = {-Stealth[inset=0pt, angle=25:5pt]}] (t0) -- (tn) node[above]{$t$};

%     % arrows 1
%     \coordinate (a1) at (0, 2);
%     \coordinate (c1) at (1.9, 1.7);
%     \draw[arrows = {-Stealth[inset=0pt, angle=25:5pt]}] (a1) -- (c1);
%     \draw[thin, dotted](a1) -- (0, 0.5);

%     \coordinate (a2) at (0,1.5);
%     \coordinate (c2_) at (1.9, 1.2);
%     \coordinate (c2) at (1.9, 1);
%     %\draw[dotted, thin, fill=gray!50] (a2)--(c2)--(c2_)--(a2);
%     \draw[arrows = {-Stealth[inset=0pt, angle=25:5pt]}] (a2) -- (c2_);

%     \coordinate (text1) at (1, 0);
%     \node[text width=2cm] at (text1){time: \cmark \\ phase: \cmark};

%     % arrows 2
%     \coordinate (a3) at (3,2);
%     \coordinate (c3) at (4.9,1.7);
%     \draw[arrows = {-Stealth[inset=0pt, angle=25:5pt]}] (a3) -- (c3);
%     \draw[thin, dotted](a3) -- (3, 1.5);

%     \coordinate (a4) at (3, 1.5);
%     \coordinate (a4_) at (3.2, 1.5);
%     \coordinate (c4) at (5.1, 1.2);
%     \draw[dotted, thin, fill=gray!50] (a4_)--(a4)--(3, 0.5)--(3.2, 0.5)--(a4_);
%     \draw[arrows = {-Stealth[inset=0pt, angle=25:5pt]}] (a4_) -- (c4);

%     \coordinate (text2) at (4, 0);
%     \node[text width=2cm] at (text2){time: \xmark \\ phase: \cmark};

%     % arrows 3
%     \coordinate (a5) at (6,2);
%     \coordinate (c5) at (7.9,1.7);
%     \draw[arrows = {-Stealth[inset=0pt, angle=25:5pt]}] (a5) -- (c5);
%     \draw[thin, dotted](a5) -- (6, 0.5);

%     \coordinate (a6) at (6,1.5);
%     \coordinate (c6_) at (7.9, 1.2);
%     \coordinate (c6) at (7.9, 1);
%     \draw[dotted, thin, fill=gray!50] (a6)--(c6)--(c6_)--(a6);
%     \draw[arrows = {-Stealth[inset=0pt, angle=25:5pt]}] (a6) -- (c6);

%     \coordinate (text3) at (7, 0);
%     \node[text width=2cm] at (text3){time: \cmark \\ phase: \xmark};
%   \end{tikzpicture}
% \end{figure}

Smart meter measurements are \textbf{local}:

% \begin{figure}[H]
\begin{tikzpicture}
\draw[fill] (0,0) circle [radius=0.1] node[above, black]{$v_{0}$};
\draw[thick] (0,0) -- (9,0);
\draw[arrows = {-Stealth[inset=0pt, angle=30:10pt]}, thick] (0,0) -- (1.5,0)  node[below]{$j_{1}$};
\draw[fill] (4,0) circle [radius=0.1] node[above]{$v_{1}, i_{1}$};
\draw[arrows = {-Stealth[inset=0pt, angle=30:10pt]}, thick] (4,0) -- (5.5,0)  node[below, red]{$j_{2}$};
\draw[arrows = {-Stealth[fill=none, inset=0pt, angle=90:10pt]}, thick] (4,0) -- (4,-1);
%\draw[arrows = {-Stealth[inset=0pt, angle=30:10pt]}, thick] (4,0) -- (4,-0.5) node[left]{$\mathbf{I}_{2}$};
\draw[fill] (8,0) circle [radius=0.1] node[above]{$v_{2}, i_{2}$};
\draw[arrows = {-Stealth[fill=none, inset=0pt, angle=90:10pt]}, thick] (8,0) -- (8,-1);
%\draw[arrows = {-Stealth[inset=0pt, angle=30:10pt]}, thick] (8,0) -- (8,-0.5) node[left]{$\mathbf{I}_{3}$};
\draw[dotted, thick] (9,0) -- (11,0);
\draw[thick] (11,0) -- (12,0);
\draw[fill] (12,0) circle [radius=0.1] node[above]{$v_{N}, i_{N}$};
\draw[arrows = {-Stealth[fill=none, inset=0pt, angle=90:10pt]}, thick] (12,0) -- (12,-1);
%\draw[arrows = {-Stealth[inset=0pt, angle=30:10pt]}, thick] (12,0) -- (12,-0.5) node[left]{$\mathbf{I}_{N}$};
\end{tikzpicture}
% \end{figure}
%\only<2-5>{
\uncover<2-5>{Phasor representation:}
\begin{columns}[t]
\uncover<2-5>{
\begin{column}{0.25\textwidth}
\centering
%$v_1 = (V_1, 0)$

%\uncover<1>{$j_1 = (J_1, \mathbin{\color{black}\angle j_1})$}
\begin{tikzpicture}[baseline]

\coordinate (a) at (0,0);
\coordinate (b) at (2,0);

\draw[arrows = {-Stealth[inset=0pt, angle=30:10pt]}] (a)--(b) node[below right]{$v_0$};
\draw[thin, dotted](b) -- (15, 0);
\end{tikzpicture}
\end{column}
}
\uncover<3-5>{
\begin{column}{0.25\textwidth}
\centering
% $v_1 = (V_1, \mathbin{\color{red}\angle v_1})$

% $i_1 = (I_1, \mathbin{\color{red}\angle i_1})$
\begin{tikzpicture}[baseline]

\coordinate (a) at (0,0);
\coordinate (b) at (2,0);
\coordinate (c) at (1.9,-0.5);
\coordinate (d) at (1.2,-0.7);

%\draw[dotted] (a) -- (b);
\draw[arrows = {-Stealth[inset=0pt, angle=30:10pt]}] (a) -- (c) node[below right]{$v_1$};
\pic[draw,-,angle eccentricity=1.15, angle radius=1.5cm, red, ultra thick] {angle=c--a--b};
\draw[arrows = {-Stealth[inset=0pt, angle=30:10pt]}] (a) -- (d) node[midway, below]{$i_1$};
%\pic["$\theta_1$", draw,-,angle eccentricity=1.4, angle radius=1cm] {angle=d--a--c};
\pic[draw,-,angle eccentricity=1.15, angle radius=0.5cm, red, ultra thick] {angle=d--a--b};
\end{tikzpicture}
\end{column}
}

\uncover<4-5>{
\begin{column}{0.25\textwidth}
\centering
% $v_2 = (V_2, \mathbin{\color{red}\angle v_2})$

% $i_2 = (I_2, \mathbin{\color{red}\angle i_2})$
\begin{tikzpicture}[baseline]

\coordinate (a) at (0,0);
\coordinate (b) at (2,0);
\coordinate (c) at (1.8,-0.8);
\coordinate (d) at (1.2,-0.9);

%\draw[dotted] (a) -- (b);
\draw[arrows = {-Stealth[inset=0pt, angle=30:10pt]}] (a) -- (c) node[right]{$v_2$};
\pic[draw,-,angle eccentricity=1.15, angle radius=1.5cm, red, ultra thick] {angle=c--a--b};
\draw[arrows = {-Stealth[inset=0pt, angle=30:10pt]}] (a) -- (d) node[midway, below]{$i_2$};
%\pic["$\theta_2$", draw,-,angle eccentricity=1.4, angle radius=1cm] {angle=d--a--c};
\pic[draw,-,angle eccentricity=1.15, angle radius=0.5cm, red, ultra thick] {angle=d--a--b};
\end{tikzpicture}
\end{column}
}

\uncover<5-5>{
\begin{column}{0.25\textwidth}
\centering
% $v_N = (V_N, \mathbin{\color{red}\angle v_N})$

% $i_N = (I_N, \mathbin{\color{red}\angle i_N})$
\begin{tikzpicture}[baseline]

\coordinate (a) at (0,0);
\coordinate (b) at (2,0);
\coordinate (c) at (1.7,-1);
\coordinate (d) at (1,-1.15);

%\draw[dotted] (a) -- (b);
\draw[arrows = {-Stealth[inset=0pt, angle=30:10pt]}] (a) -- (c) node[right]{$v_N$};
\pic[draw,-,angle eccentricity=1.15, angle radius=1.5cm, red, ultra thick] {angle=c--a--b};
\draw[arrows = {-Stealth[inset=0pt, angle=30:10pt]}] (a) -- (d) node[midway, below]{$i_N$};
%\pic["$\theta_N$", draw,-,angle eccentricity=1.4, angle radius=1cm] {angle=d--a--c};
\pic[draw,-,angle eccentricity=1.15, angle radius=0.5cm, red, ultra thick] {angle=d--a--b};

\end{tikzpicture}
\end{column}
}
\end{columns}
%}

\uncover<5->{
	\textbf{Note}: 
	\begin{itemize}
		\item Phase reference is chosen w.r.t. the first node and global phase angles (shown in \textcolor{red}{red}) are unknown
    	\item PMU and Micro Phasor Measurement Units ($\mu$PMUs) address this problem \cite{von2014micro}
	\end{itemize}	
}

\setbeamercolor{alerted text}{fg=black ,bg=yellow}
\end{frame}


\begin{frame}[t]{Model}

\begin{columns}
	\colorbox{gray!20}{\begin{column}{0.5\textwidth}
		\textbf{Backward model}
		\begin{equation*}
		  \begin{split}
		  v_{n-1}e^{i\Delta_{n}} - v_{n} = j_{n}z_{n}, \\
		  j_{n-1} = i_{n-1} + j_{n}e^{-i\Delta_{n}}
		  \end{split}
		\end{equation*}

		\textbf{Idea:} Compute impedances iteratively starting from the last node.
	\end{column}}
	\begin{column}{0.5\textwidth}
		%\begin{figure}[H]
		\begin{tikzpicture}
			\draw[thin, dotted] (-2, 0) -- (0, 0);
			\draw[arrows = {-Stealth[inset=0pt, angle=30:10pt]}, dotted] (-2, 0) -- (-0.5,0)  node[below left]{$j_{n-1}$};
			\draw[fill] (0,0) circle [radius=0.1] node[above, black]{$v_{n-1}$};
			\draw[arrows = {-Stealth[fill=none, inset=0pt, angle=90:10pt]}, thick] (0,0) -- node[right]{$i_{n-1}$}(0, -1);
			\draw[thick] (0,0) -- (4,0);
			\draw[arrows = {-Stealth[inset=0pt, angle=30:10pt]}, thick] (0, 0) -- (1.5,0)  node[below]{$j_{n}$};
			\draw[fill] (4,0) circle [radius=0.1] node[above]{$v_{n}$};
			\draw[arrows = {-Stealth[fill=none, inset=0pt, angle=90:10pt]}, thick] (4, 0) -- node[right]{$i_{n}$}(4, -1);

			\coordinate (a) at (-1, -1.5);
			\coordinate (b) at (1, -1.5);
			\coordinate (c) at (0.9, -1.1);

			\draw[dotted] (a) -- node[below right]{$\Delta_{n}$} (b);
			\draw[arrows = {-Stealth[inset=0pt, angle=30:10pt]}] (a) -- (c) node[below right]{$v_{n-1}$};
			\pic[draw,-,angle eccentricity=1.35, angle radius=1.4cm, thick] {angle=b--a--c};
			%\pic[draw,-,angle eccentricity=1.15, angle radius=1.5cm, red, ultra thick] {angle=c--a--b};
			%\draw[arrows = {-Stealth[inset=0pt, angle=30:10pt]}] (a) -- (d) node[midway, below]{$i_1$};
			%\pic["$\theta_1$", draw,-,angle eccentricity=1.4, angle radius=1cm] {angle=d--a--c};
			%\pic[draw,-,angle eccentricity=1.15, angle radius=0.5cm, red, ultra thick] {angle=d--a--b};

			\coordinate (a1) at (3, -1.5);
			\coordinate (b1) at (5, -1.5);
			\coordinate (c_shifted) at (4.9, -1.8);
			\coordinate (c1) at (4.6, -2.4);

			%\draw[dotted] (a1) -- (b1);
			%\draw[arrows = {-Stealth[inset=0pt, angle=30:10pt]}, dashed] (a1) -- (c_shifted) node[right]{$v_{n-1}$};
			\draw[arrows = {-Stealth[inset=0pt, angle=30:10pt]}] (a1) -- (b1) node[below right]{$v_n$};
			%\pic["$\Delta_{n}$", draw,-,angle eccentricity=1.35, angle radius=1.4cm, thick] {angle=c1--a1--c_shifted};

		\end{tikzpicture}
		%\end{figure}
	\end{column}
\end{columns}

\hfill

\uncover<2->{
	\textbf{Benefits}:
	\begin{itemize}
		\item Low complexity: $O(MN)$ vs $O((M+2N)^2MN)$ in conventional case \cite{peppanen2015distribution}.
		\item Backward model is a better conditioned set of equations than \textbf{power flow equations} for impedance identification.
		\item Backward model can be used to solve any \textbf{tree} type network.
	\end{itemize}
}
\uncover<3->{
	\textbf{Note}:
	\begin{itemize}
		\item Equations are derived in terms of phase increments
		\item Backward model is equivalent to \textbf{power flow equations}.
	\end{itemize}	
}

\end{frame}



\end{document}